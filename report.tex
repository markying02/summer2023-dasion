\documentclass{article}
\usepackage[utf8]{inputenc}
\usepackage{multirow}
\usepackage{float}
\usepackage{geometry}
\usepackage{longtable}
\geometry{a4paper,left=3cm,top=3cm,right=3cm,bottom=3cm}

\title{Speech-Language Evaluation Report}
\author{Dr. Smith}
\date{August 9, 2023}

\begin{document}

\maketitle

\section{Introduction}

This report presents the results of the speech-language evaluation conducted on Nick Wong, a 7-year-old male, on August 9, 2023. The evaluation aimed to assess various aspects of Nick's speech and language skills including speech production, grammar, vocabulary, and speech rate. The findings of the evaluation will be discussed in detail, followed by recommendations for intervention.

\section{Findings}

\subsection{Speech Production}

To evaluate Nick's speech production, several measures were analyzed, including word count, average word length, total syllables, and the number of phoneme instances.

\begin{table}[H]
\centering
\caption{Speech Production Statistics}
\begin{tabular}{|c|c|}
\hline
\textbf{Statistic} & \textbf{Value} \\ \hline
Word Count & 71 \\ \hline
Average Word Length & 3.464788732394366 \\ \hline
Total Syllables & 86 \\ \hline
Number of Phoneme Instances & see Appendix A \\ \hline
\end{tabular}
\end{table}

The word count indicates that Nick used a total of 71 words during the speech recording. This measure provides information about the length or verbosity of Nick's speech. 

The average word length of 3.46 suggests that Nick's words, on average, consisted of approximately 3.46 letters. This measure gives insights into the complexity of Nick's spoken words.

The total syllables count of 86 reflects the overall number of syllables pronounced by Nick during the recording. This measure provides information about the phonological complexity of Nick's speech, as longer words tend to have more syllables.

The number of phoneme instances (see Appendix A for the complete list) corresponds to the number of occurrences of each phoneme in Nick's speech. Phonemes are the smallest units of sound in a language, and analyzing their production can reveal any potential speech sound errors or patterns in Nick's speech.

\subsection{Grammar and Vocabulary}

Nick's grammar and vocabulary were assessed through measuring the number of unique words, average instances of each word, the number of question words, entropy, and the measure of lexical textual diversity score.

\begin{table}[H]
\centering
\caption{Grammar and Vocabulary Statistics}
\begin{tabular}{|c|c|}
\hline
\textbf{Statistic} & \textbf{Value} \\ \hline
Number of Unique Words & 33 \\ \hline
Average Instances of Each Word & 2.1515 \\ \hline
Number of Question Words & 4 \\ \hline
Entropy & 4.1455 \\ \hline
Measure of Lexical Textual Diversity Score & 12.8490 \\ \hline
\end{tabular}
\end{table}

The number of unique words indicates that out of the 71 words used by Nick, there were 33 distinct words. This measure reflects the richness of Nick's vocabulary and the variety of words he used during the speech recording.

The average instances of each word (2.1515) suggests that, on average, each word was repeated approximately 2.1515 times. This measure provides information about the diversity and range of Nick's vocabulary. 

The number of question words (4) indicates how many interrogative words were used by Nick during the recording. This measure demonstrates Nick's understanding and use of question words in his speech.

Entropy (4.1455) is a measure of how evenly the words are distributed in the speech sample. Higher entropy values suggest greater linguistic complexity, variability, and diversity in word usage.

The measure of lexical textual diversity score (12.8490) provides an overall index of the diversity and richness of Nick's vocabulary. This measure considers factors such as the number of unique words, their frequencies, and the length of the speech sample.

\subsection{Speech Rate}

The speech rate was assessed through measuring the number of sentences, number of syllables, number of pauses, rate of speech, articulation rate, speaking duration, original duration, and balance.

\begin{table}[H]
\centering
\caption{Speech Rate Statistics}
\begin{tabular}{|c|c|}
\hline
\textbf{Statistic} & \textbf{Value} \\ \hline
Number of Sentences & 16 \\ \hline
Number of Syllables & 117 \\ \hline
Number of Pauses & 36 \\ \hline
Rate of Speech & 1 \\ \hline
Articulation Rate & 5 \\ \hline
Speaking Duration (seconds) & 24.5 \\ \hline
Original Duration (seconds) & 86.8 \\ \hline
Balance & 0.3 \\ \hline
\end{tabular}
\end{table}

The number of sentences (16) reflects the total number of complete sentences or independent clauses used by Nick. This measure provides information about the syntactic complexity and structure of Nick's speech.

The number of syllables (117) reflects the overall phonetic complexity of Nick's speech, as longer sentences tend to have more syllables. This measure is also related to the pace and rhythm of Nick's speech.

The number of pauses (36) indicates how many breaks or hesitations Nick took during the speech recording. Pauses can provide insights into Nick's speech fluency and rate of speaking.

The rate of speech (1) indicates the speed at which Nick spoke during the recording. A rate of 1 suggests that Nick spoke at a moderate pace.

The articulation rate (5) provides information about the efficiency and clarity of Nick's speech production. This measure is calculated by dividing the total number of syllables by the speaking duration (in seconds).

The speaking duration (24.5 seconds) reflects the total time Nick spent speaking during the recording. This measure indicates the length or duration of Nick's speech.

The original duration (86.8 seconds) represents the total duration of the speech recording, including both speaking and non-speaking sections. This measure provides insights into Nick's overall ability to sustain attention and stay engaged in verbal communication.

The balance (0.3) is a measure that compares the speaking duration with the original duration, indicating the proportion of speech in relation to the total recording time. A balance of 0.3 suggests that approximately 30\% of the recording consisted of Nick's speech.

\subsection{Pitch}

The pitch was assessed through measuring the fundamental frequency (f0) mean, f0 standard deviation (std), f0 median, f0 minimum, f0 maximum, f0 25th percentile (quantile25), and f0 75th percentile (quantile75).

\begin{table}[H]
\centering
\caption{Pitch Statistics}
\begin{tabular}{|c|c|}
\hline
\textbf{Statistic} & \textbf{Value} \\ \hline
f0 Mean & 255.28 Hz \\ \hline
f0 Standard Deviation & 75.06 Hz \\ \hline
f0 Median & 252.8 Hz \\ \hline
f0 Minimum & 80 Hz \\ \hline
f0 Maximum & 421 Hz \\ \hline
f0 25th Percentile & 208 Hz \\ \hline
f0 75th Percentile & 309 Hz \\ \hline
\end{tabular}
\end{table}

The f0 mean (255.28 Hz) represents the average fundamental frequency of Nick's voice. This measure provides insights into the pitch range and intonation patterns exhibited by Nick during the speech recording.

The f0 standard deviation (75.06 Hz) indicates the variability or dispersion of Nick's fundamental frequency values. A higher standard deviation suggests greater pitch variation, while a lower standard deviation suggests more monotony in pitch.

The f0 median (252.8 Hz) represents the middle value of the fundamental frequency distribution. This measure indicates the central tendency of Nick's pitch.

The f0 minimum (80 Hz) and f0 maximum (421 Hz) represent the lowest and highest fundamental frequencies exhibited by Nick, respectively. These measures provide information about the pitch range and any outliers in Nick's voice.

The f0 25th percentile (208 Hz) and f0 75th percentile (309 Hz) reflect the values below which 25\% and 75\% of Nick's fundamental frequency values fall, respectively. These measures further contribute to our understanding of the pitch range and distribution in Nick's voice.

\subsection{Grammatical Issues}

The number of grammatical issues was assessed as an indicator of Nick's grasp of grammar and sentence structure. The evaluation identified a total of 13 grammatical issues during the speech recording.

\subsection{Noun Phrases}

The number of noun phrases (noun + adjectives) provides insights into Nick's ability to form and use noun phrases in his speech. The evaluation detected 7 distinct noun phrases used by Nick.

\subsection{Polarity and Subjectivity}

Polarity and subjectivity are measures that provide information about the emotional tone and perspective expressed in Nick's speech. The evaluation yielded a polarity score of -0.1333 and a subjectivity score of 0.8657. A negative polarity score suggests a slightly negative sentiment, while a subjectivity score close to 1 indicates a more subjective perspective.

\subsection{Overall Evaluation}

Overall, Nick demonstrated age-appropriate speech production, with a diverse vocabulary and linguistic complexity. However, a moderate number of grammatical issues were identified. Nick's speech rate was within normal limits, with clear articulation and average pitch range. The evaluation indicated a slightly negative polarity score and a subjectivity score indicating some subjective expressions.

\section{Recommendation}

Based on the findings of the speech-language evaluation, the following recommendations are proposed:

\begin{itemize}
    \item \textbf{Speech Therapy:} Given the presence of grammatical issues, it is recommended that Nick engages in regular speech therapy sessions to address and improve his understanding and use of grammar rules and sentence structures.
    
    \item \textbf{Vocabulary Enhancement:} Although Nick demonstrated a diverse vocabulary, efforts can still be made to enrich his lexicon. It is suggested that Nick engages in activities that promote exposure to new words, such as reading books, engaging in conversations on various topics, and playing word-based games.
    
    \item \textbf{Fluency Building:} To enhance speech fluency, Nick can practice techniques that reduce pauses and hesitations, such as slow and deliberate speaking, increasing breath support, and utilizing appropriate pacing in verbal communication.
    
    \item \textbf{Intonation Training:} Based on the pitch analysis, Nick may benefit from intonation training to improve the range and expressiveness of his voice. This can be achieved through exercises that focus on pitch variation, imitation of different intonation patterns, and awareness of rising and falling pitch contours.
    
    \item \textbf{Emotional Expression in Communication:} Given the slightly negative polarity score, Nick can engage in activities that promote understanding and expression of a wider range of emotions in his speech. This can be achieved through storytelling, role-playing, and conversations that encourage reflection on personal experiences and emotions.
\end{itemize}

These recommendations aim to address the identified areas of concern and further enhance Nick's speech and language skills. Regular evaluation and progress monitoring are recommended to track Nick's development and adjust intervention goals accordingly.

\vspace{2cm}

\section*{Appendix A: Phoneme Instances}

The following table presents the number of instances for each phoneme observed in Nick's speech recording.

\begin{table}[H]
\centering
\caption{Phoneme Instances}
\begin{tabular}{|c|c|}
\hline
\textbf{Phoneme} & \textbf{Number of Instances} \\ \hline
N & 22 \\ \hline
T & 21 \\ \hline
AH1 & 13 \\ \hline
S & 12 \\ \hline
D & 11 \\ \hline
UW1 & 10 \\ \hline
IY0 & 10 \\ \hline
F & 10 \\ \hline
Y & 9 \\ \hline
IH1 & 9 \\ \hline
AA1 & 9 \\ \hline
OW1 & 7 \\ \hline
AH0 & 7 \\ \hline
W & 6 \\ \hline
EY1 & 5 \\ \hline
P & 4 \\ \hline
G & 4 \\ \hline
DH & 4 \\ \hline
M & 3 \\ \hline
EH1 & 3 \\ \hline
B & 3 \\ \hline
AY1 & 3 \\ \hline
AE1 & 3 \\ \hline
Z & 2 \\ \hline
AW1 & 2 \\ \hline
R & 1 \\ \hline
OY1 & 1 \\ \hline
NG & 1 \\ \hline
L & 1 \\ \hline
K & 1 \\ \hline
HH & 1 \\ \hline
ER1 & 1 \\ \hline
ER0 & 1 \\ \hline
\end{tabular}
\end{table}

\end{document}